% ********** Rozdział 1 **********
\chapter{Opis założeń projektu}
\section{Cele projetu}
%\subsection{Tytuł pierwszego podpunktu}

Celem głównym projektu jest stworzenie kompleksowej aplikacji desktopowej, mającej na celu usprawnienie procesów zarządzania w środowisku szpitalnym. Aplikacja \textquotedbl Szpital+\textquotedbl{} ma usprawnić gromadzenie danych medycznych oraz zwiększyć efektywność komunikacji w placówce medycznej. Aplikacja posiada 4 klienta, funkcjonalność których się różni:
\renewcommand{\labelenumi}{\alph{enumi})}
\begin{enumerate}
    \item{Klient Głównego Kierownika}
    \item{Klient Kierownika Działu} 
    \item{Klient Recepcjonisty}
    \item{Klient Lekarza}
\end{enumerate}

\section{Wymagania funkcjonalne}

\begin{itemize}
    \item Logowanie do systemu.
    \item Przegląd informacji o siebie lub innym pracowniku.
    \item Możliwość dodawać, usuwać lub zmieniać wizyty.
    \item Możliwość dodawać lub zwalniać pracowników.
    \item Możliwość wylogowania.
    \item Dodanie lub usuwanie zapisów w książkach pacjentów.
    \item Dodanie i usunięcie pacjentów.
\end{itemize}

\section{Wymagania niefunkcjonalne}

\begin{itemize}
    \item \textbf{Łatwy w użyciu i przejrzysty interfejs użytkownika:} interfejs musi odpowiadać nowoczesnym stylom budowania aplikacji GUI.
    \item \textbf{Walidacja danych:} program musi sprawdzać poprawność wpisanych przez użytkownika danych.
    \item \textbf{Wyświetlenie komunikatów:} przy wpisaniu błędnych danych aplikacja ma powiadomić użytkownika o pojawiającym się błędzie.
    \item \textbf{Prawidłowo działająca baza danych:} baza danych ma pozwalać na operacje CRUD i być logicznie zdefiniowana.
\end{itemize}

% ********** Koniec rozdziału **********

\begin{flushleft}
\section{\LARGE{Wymagania funkcjonalne}}
\end{flushleft}

\begin{flushleft}
    \subsection{\large{Ogólne funkcjonalności pracowników\label{subsec:gl_kier}}}
    \begin{itemize}
        \item{Przegląd informacji o siebie (sekcja ~\ref{subsec:przeg_inf})}
        \item{Wylogowanie się}
    \end{itemize}
\end{flushleft}

%\renewcommand{\labelitemi}{\textsc{-}}
\begin{flushleft}
    \subsection{\large{Klient Głównego Kierownika\label{subsec:gl_kier}}}
    \begin{itemize}
        \item{Wyświetlenie listy pracowników (sekcja ~\ref{subsec:wyst_prac})}
    \end{itemize}
\end{flushleft}

\begin{flushleft}
    \subsection{\large{Klient Kierownika Działu\label{subsec:kier}}}
    \begin{itemize}
        \item{Wyświetlenie listy lekarzy (sekcja ~\ref{subsec:wyst_lekrz})}
    \end{itemize}
\end{flushleft}

\begin{flushleft}
    \subsection{\large{Klient Recepcjonisty\label{subsec:recep}}}
    \begin{itemize}
        \item{Wyświetlenie listy lekarzy (sekcja ~\ref{subsec:wyst_lekrz})}
        \item{Funkcja dodawania pacjentów (sekcja ~\ref{subsec:dodn_pacjn})}
        % \item{Przegląd harmonogramu pacjenta (sekcja ~\ref{subsec:wyst_harm_klnt})}
        \item{Funkcja przeglądu wszystkich wizyt (sekcja ~\ref{subsec:przeg_wiz})}
    \end{itemize}
\end{flushleft}

\begin{flushleft}
    \subsection{\large{Klient Lekarza\label{subsec:lekarz}}}
    \begin{itemize}
        \item{Funkcja ustalenia swoich wizytów (sekcja ~\ref{subsec:ust_wiz})}
        \item{Przegląd swojego harmonogramu (sekcja ~\ref{subsec:wyst_harm_lekrz})}
        \item{Przegląd informacji o pacjencie (sekcja ~\ref{subsec:przeg_inf})}
        \item{Przegląd ksiązek lekarskich pacjentów (sekcja ~\ref{subsec:przeg_ks_lekr})}
    \end{itemize}
\end{flushleft}

\begin{flushleft}
    \subsection{\Large{Opis funkcjonalności}}
    \subsubsection{\large{Przegląd informacji}\label{subsec:przeg_inf}}
    Wyświetla informacje:
    \begin{itemize}
        \item{Dla pracownika: id pracownika, imię, nazwisko, dział szpitalu, specjalność, pesel, adres zamieszkania, datę urodzenia, miejscowość, numer telefonu, datę zatrudnienia, wynagrodzenie, email.}
        \item{Dla pacjenta: id pracjenta, imię, nazwisko, pesel, adres zamieszkania, datę urodzenia, miejscowość, numer telefonu}\label{item:info}
    \end{itemize}
\end{flushleft}

\begin{flushleft}
    \subsubsection{\large{Wyświetlenie listy pracowników}\label{subsec:wyst_prac}}
    Pozwala głównemu kierownikowi wyświetlić listę pracowników. Lista zawiera następujące funkcje:
    \begin{itemize}
        \item Funkcja przęglądu informacji o pracowniku (sekcja ~\ref{subsec:przeg_inf})
        \item Funkcja ustalenia pensji pracownika\label{item:pens_prac}
        \item Funkcja zwolnienia pracownika (usuwa pracownika z bazy danych)
        \item Funkcja dodawania nowego pracownika (sekcja ~\ref{subsec:dodn_prac})
        \item Funkcja wyświetlania harmonogramu lekarza
    \end{itemize}
\end{flushleft}

\begin{flushleft}
    \subsubsection{\large{Wyświetlenie listy lekarzy }\label{subsec:wyst_lekrz}}
    Pozwala kierownikowi wyświetlić listę lekarzy swojego działu. Lista zawiera następujące funkcje:
    \begin{itemize}
        \item Funkcja przęglądu informacji o lekarzu (sekcja ~\ref{subsec:przeg_inf})
        \item Funkcja zwolnienia lekarza (usuwa lekarza z bazy danych)
        \item Funkcja dodawania nowego lekarza (sekcja ~\ref{subsec:dodn_prac})
        \item Funkcja wyświetlania harmonogramu lekarza (sekcja ~\ref{subsec:wyst_harm_lekrz})
        \item Funkcja ustalenia pensji lekarza
    \end{itemize}
\end{flushleft}

\begin{flushleft}
    \subsubsection{\large{Wyświetlenie harmonogramu lekarzy }\label{subsec:wyst_harm_lekrz}}
    Wyświetla tablicę z wizytami lekarza.\\
    \hspace{5mm}Dla recepcjonisty funkcja pozwala na ustalenie wizytów (sekcja ~\ref{subsec:ust_wiz}), usunięcie wizytów (sekcja ~\ref{subsec:usun_wiz}), wyświetlenie informacji o pacjencie i o doktorze (sekcja ~\ref{subsec:przeg_inf}).\\
    \hspace{5mm}Dla lekarza pozwala na wyżej wymienione funkcję dodatkowe dla recepcjonisty tylko dla siebie.
\end{flushleft}

\begin{flushleft}
    \subsubsection{\large{Funkcja ustalenia wizyty }\label{subsec:ust_wiz}}
    Pozwala na dodanie do tabeli bazy danych z planem wizytów nowych rekordów z informacjami: id wizyty, data i godzina wizyty, id lekarza, id pacjenta, numer gabinetu i notatki do wizyty.
\end{flushleft}

\begin{flushleft}
    \subsubsection{\large{Funkcja usunięcia wizyty }\label{subsec:usun_wiz}}
    Usuwa rekord(wizytę) z tabeli bazy danych z planem wizytów o wskazanym id.
\end{flushleft}

\begin{flushleft}
    \subsubsection{\large{Funkcja dodawania pacjentów }\label{subsec:dodn_pacjn}}
    Dodaje do tabeli bazy danych z pacjentami nowego pacjenta o informacjach wymienionych w sekcji ~\ref{subsec:przeg_inf} punkt ~\hyperref[item:info]{2}
\end{flushleft}

\begin{flushleft}
    \subsubsection{\large{Funkcja przeglądu harmonogramu klienta}\label{subsec:wyst_harm_klnt}}
    Wyświetla tablicę z wizytami pacjenta.
\end{flushleft}

\begin{flushleft}
    \subsubsection{\large{Funkcja przeglądu ksiązek lekarskich}\label{subsec:przeg_ks_lekr}}
    Wyświetla tablicę baz danych ksiązki lekarskiej pacjenta posortowanej po datach nonatek.\\
    Dla lekarza dostępna funkcja dodawania notatek do ksiązki (sekcja ~\ref{subsec:dodn_notat_do_ks_lekr}).
\end{flushleft}

\begin{flushleft}
    \subsubsection{\large{Funkcja dodawania notatek i recepty do książek lekarskich pacjentów}\label{subsec:dodn_notat_do_ks_lekr}}
    Dodaje nowe notatki do tabeli bazy danych z notatkami wszystkich pacjentów.
\end{flushleft}

\begin{flushleft}
    \subsubsection{\large{Funkcja przeglądu wszystkich wizyt}\label{subsec:przeg_wiz}}
    Wyświetla tablicę czasową z wszystkimi wizytami. Wizyty przedstawione jako bloki. \\
    Dla recepcjonisty dodaje funkcję usunięcia wizyty i dodawania wizyty (sekcja  ~\ref{subsec:usun_wiz} i ~\ref{subsec:ust_wiz}).
\end{flushleft}

\begin{flushleft}
    \subsubsection{\large{Funkcja dodawania nowego pracownika}\label{subsec:dodn_prac}}
    Dodaje pracowników do tabeli z pracownikami w bazie danych z informacją o pracowniku z sekcji ~\ref{subsec:przeg_inf}. Główny kierownik może dodawać: kierowników, recepcjonistów i lekarzy. Kierownicy mogą dodawać tylko lekarzy.
\end{flushleft}


\begin{flushleft}
\subsection{\Large{Ograniczenia i zabronione operacje}}
\end{flushleft}

\begin{flushleft}
    \subsubsection{\large{Wprowadzanie danych: }}
    \renewcommand{\labelenumii}{\alph{enumii})}
    \begin{enumerate}
        \item{Dla finkcji ustalenia wizytów (sekcja ~\ref{subsec:ust_wiz}):
        \begin{enumerate}
            \item{Nie można wprowadzać id wizyty(generuje się automatycznie);}
            \item{Data i godzina wizyty muszą być względnie z formatem "YYYY-MM-DD HH:MI:SS";}
            \item{Id lekarza nie może być mniejsze od 1 i większe od liczby lekarzy;}
            \item{Id pacjenta nie może być mniejsze od 1 i większe od liczby pacjentów;}
            \item{Numer gabinetu nie może być mniejsze od 1 i większe od liczby gabinetów;}
        \end{enumerate}}
        \item{Dla funkcji dodawania pacjentów (sekcja ~\ref{subsec:dodn_pacjn}):\label{item:zabr_dodn_prac}
        \begin{enumerate}
            \item{Nie można wprowadzać id pacjenta(generuje się automatycznie);\label{item:zabr_dodn_prac_a}}
            \item{Imię musi się składać z liter i nie być mniejsze od 2 i większe od 30 liter;}
            \item{Nazwisko musi się składać z liter i nie być mniejsze od 2 i większe od 50 liter;}
            \item{Data urodzenia musi być względnie z formatem "YYYY-MM-DD";}
            \item{Numer pesel musi się skłądać z 11 cyfr;}
            \item{Numer telefonu musi być względnie z formatem "+48 xxx xxx xxx";}
            \item{Adres email musi być względnie z formatem "username@pracownik.szp.pl";}
        \end{enumerate}}
        \item{Dla funkcji ustalenia pensji pracowników (sekcja ~\ref{subsec:wyst_prac} punkt \hyperref[item:pens_prac]{2}):\\
        \hspace{2.5mm}Pensja musi być liczbą rzeczywistą większą od 0.}
        \item{Dla funkcji dodawania nowego pracownika (sekcja ~\ref{subsec:dodn_prac})
        \begin{enumerate}
            \item{Są te same ograniczenia jak dla funkcji dodawania pacjentów (sekcja ~\ref{item:zabr_dodn_prac})}
            \item{Data zatrudnienia musi być względnie z formatem "YYYY-MM-DD";}
        \end{enumerate}}
        \item{Dla funkcji dodawania notatek i recepty do książek lekarskich pacjentów (sekcja ~\ref{subsec:dodn_notat_do_ks_lekr})
        \begin{enumerate}
            \item{Id pacjenta musi być większy od 0 i mniejszy od liczby pacjentów;}
            \item{Id lekarza musi być większy od 0 i mniejszy od liczby pacjentów;}
            \item{Data notatki musi być względnie z formatem "YYYY-MM-DD";}
        \end{enumerate}}
    \end{enumerate}
\end{flushleft}